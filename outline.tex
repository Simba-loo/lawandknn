%
% This is the LaTeX template file for lecture notes for CS294-8,
% Computational Biology for Computer Scientists.  When preparing 
% LaTeX notes for this class, please use this template.
%
% To familiarize yourself with this template, the body contains
% some examples of its use.  Look them over.  Then you can
% run LaTeX on this file.  After you have LaTeXed this file then
% you can look over the result either by printing it out with
% dvips or using xdvi.
%
% This template is based on the template for Prof. Sinclair's CS 270.

\documentclass{article}
\usepackage{graphics}
\setlength{\oddsidemargin}{0.25 in}
\setlength{\evensidemargin}{-0.25 in}
\setlength{\topmargin}{-0.6 in}
\setlength{\textwidth}{6.5 in}
\setlength{\textheight}{8.5 in}
\setlength{\headsep}{0.75 in}
\setlength{\parindent}{0 in}
\setlength{\parskip}{0.1 in}

\usepackage[ruled]{algorithm2e} % For algorithms
\renewcommand{\algorithmcfname}{ALGORITHM}
\SetAlFnt{\small}
\SetAlCapFnt{\small}
\SetAlCapNameFnt{\small}
\SetAlCapHSkip{0pt}
\IncMargin{-\parindent}


%\documentclass{llncs}
%\usepackage{llncsdoc}

%% ==== packages =====
%%\usepackage{latexsym}
\usepackage{amsmath,amssymb,enumitem,dsfont,bm,url,graphicx,comment,mathtools}
%\usepackage{cite}
%\let\proof\relax
%\let\endproof\relax
\usepackage{amsthm}
%
%\usepackage{color}
%
%\usepackage[colorinlistoftodos]{todonotes}
\usepackage{algorithmic}% http://ctan.org/pkg/algorithms
%\renewcommand{\algorithmicrequire}{\textbf{Inputs:}}
%\renewcommand{\algorithmicensure}{\textbf{Outputs:}}
%\usepackage{tikz}
%\setlength{\intextsep}{1\baselineskip}
%
%
\def\EE{{\mathbb{E}}}
\def\PP{{\mathbb{P}}}
\DeclarePairedDelimiter{\abs}{\lvert}{\rvert} %
\DeclarePairedDelimiter{\brk}{[}{]}
\DeclarePairedDelimiter{\crl}{\{}{\}}
\DeclarePairedDelimiter{\prn}{(}{)}

\newcommand{\nc}{\newcommand}
\nc{\nt}{\newtheorem}
\nt{theorem}{Theorem}[section]
\nt{cor}[theorem]{Corollary}
\nt{prop}[theorem]{Proposition}
\nt{lemma}[theorem]{Lemma}
\nt{conjecture}[theorem]{Conjecture}
\nt{defn}[theorem]{Definition}
\begin{document}
%FILL IN THE RIGHT INFO.
%\lecture{**LECTURE-NUMBER**}{**DATE**}{**LECTURER**}{**SCRIBE**}
\title{Law and KNN}
\date{\vspace{-5ex}}
\maketitle
%\footnotetext{These notes are partially based on those of Nigel Mansell.}

% **** YOUR NOTES GO HERE:

% Some general latex examples and examples making use of the
% macros follow.  
%**** IN GENERAL, BE BRIEF. LONG SCRIBE NOTES, NO MATTER HOW WELL WRITTEN,
%**** ARE NEVER READ BY ANYBODY.

\section{Setting}

Let $X\subset R^n$ be a closed connected set of all possible cases, and $p(x): X\rightarrow [0,1]$ the fraction of people who believe the outcome of case $x$ should be $1$ (vs $0$); we assume $p(\cdot)$ is Lipschitz.

Cases $x_t\sim uniform[X]$ arrive at each time $t$ together with an opinion of a randomly sampled person $o_t\in \{0,1\}$, and a decision $d_t$ has to be made on a case at the time of arrival.  The quality of history of decisions $h = \{x_t,d_t\}$ is evaluated by the number of people who disagree with them:

$$L(h) = \sum_{t=1}^\infty e^{-\delta t}|p(x_t) -d_t|,$$
where $\delta$ is the discount factor. We let $L^*(h)$ be the optimal value of $L(h)$ given sequence $\{x_t\}$, that is value of $L$ when $d_t = [p(x_t)]$, where $[]$ denotes rounding to closest integer.

\section{Algorithms}

\subsection{Local precedents}

Inputs: integer $k$, real $D_{max}$.

At every step the algorithm maintains the set of precedents $S$, elements of which are case-decision tuplets $(x,d)$; $S$ is initialized with empty set. For a case $x_t$, if there are at least $k$ precedents within distance $D$ of $x_t$, the decision $d_t$ is made by the majority rule over $k$ nearest precedents to $x_t$. If there are not $k$ precedents within $D$, the decision is set to $d_t = o_t$ and a precedent $(x_t,d_t)$ is added to $S$.

\begin{theorem}
As $\delta\rightarrow 0,\ k\rightarrow \infty, D\rightarrow 0$, the loss converges to optimal $\frac{L(h)}{L^*(h)}\rightarrow 1$.
\end{theorem}

\begin{conjecture}
There exists an algorithm that uses finite memory while achieving the optimality condition above for $\delta \rightarrow ->1$.
\end{conjecture}

\subsection{Single juror}
Set $d_t=o_t$.

Generalizes to multiple jurors when there are multiple samples $o_t$ available for a single case.

\subsection{Rulebook}

Eg: As cases arrive, a decision tree is constructed and cases are settled according to that tree



\section{Questions }

\begin{enumerate}
\item When do local precedents out-compete single juror?
\item Is conjecture true?
\item What's a good algorithm when $p(x)$ changes over time?
\item (vague) how do costs of all algorithms compare as the dimension of $X$ is increased?
\item Is there a tractable game-theoretic angle here?
\end{enumerate}


\end{document}





